% @Author: UnsignedByte
% @Date:   11:12:32, 08-Dec-2020
% @Last Modified by:   UnsignedByte
% @Last Modified time: 10:37:27, 10-Dec-2020

\documentclass{article}
\usepackage{amsmath}
\usepackage{graphicx}
\graphicspath{ {./images/} }
\usepackage[%  
    colorlinks=true,
    pdfborder={0 0 0},
    linkcolor=blue
]{hyperref}
\begin{document}
\begin{titlepage}
	\vspace*{\stretch{1.0}}
	\begin{center}
		\Large\textbf{Machine Learning and Mixed Strategy Games}\\
		\large\textit{Edmund Lam, Avi Patni}
	\end{center}
	\vspace*{\stretch{2.0}}
\end{titlepage}

\section{Background}
\subsection{Mixed Strategy Games}

\begin{figure}[b]
  \caption{Example $2\times2$ payoff matrix with sample moves and resulting scores.}
  	\begin{align*}
  	&\text{Payoff Matrix} &(P_1&,P_2) &\text{Payoffs}\\
	  &M=
	  \begin{bmatrix}
		  2 & 1\\
		  3 & 0
	  \end{bmatrix}
	  &(1&,2)
	  &(1,3)
	  \end{align*}
  \label{fig:1}
\end{figure}

A Mixed Strategy Game within this paper refers to a subset of strategic games within game theory. A game is defined to have $P$ players and $M$ moves per player. Each player can play any one of their $M$ moves every round, and scores for each player are calculated by accessing the payoff matrix with the results of all their opponents. The payoff matrix is represented by a $P$-dimensional hypercube matrix with side length $M$. Players can be conceptualized as being arranged in a circle, each player is the first in their perspective, which increments for player to their right. Thus, player $P_i$'s' $(1\leq i\leq P)$ score is calculated by accessing the payoff matrix with the coordinates $(P_i, P_{i+1}, \cdots , P_P, P_1, P_2, \cdots , P_{i-1})$. Figure \ref{fig:1} shows an example of how payoffs could be calculated for each player. Players $1$ and $2$ make moves $P_1$ and $P_2$, respectively. Thus, the players will recieve scores $M_{<1,2>}=3$ and $M_{<2,1>}=3$, respectively.

This project uses 


\section{Methods and Procedure}

\end{document}