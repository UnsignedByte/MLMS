% @Author: UnsignedByte
% @Date:   11:12:32, 08-Dec-2020
% @Last Modified by:   UnsignedByte
% @Last Modified time: 10:48:36, 10-Dec-2020

\documentclass{article}
\usepackage{amsmath}
\usepackage{graphicx}
\graphicspath{ {./images/} }
\usepackage[%  
    colorlinks=true,
    pdfborder={0 0 0},
    linkcolor=blue
]{hyperref}
\begin{document}
\begin{titlepage}
	\vspace*{\stretch{1.0}}
	\begin{center}
		\Large\textbf{Machine Learning and Mixed Strategy Games}\\
		\large\textit{Edmund Lam, Avi Patni}
	\end{center}
	\vspace*{\stretch{2.0}}
\end{titlepage}

\section{Introduction}

\begin{figure}[b]
  \caption{Example $2\times2$ payoff matrix with sample moves and resulting scores.}
  	\begin{align*}
  	&\text{Payoff Matrix} &(P_1&,P_2) &\text{Payoffs}\\
	  &M=
	  \begin{bmatrix}
		  2 & 1\\
		  3 & 0
	  \end{bmatrix}
	  &(1&,2)
	  &(1,3)
	  \end{align*}
  \label{fig:1}
\end{figure}

A mixed strategy game within this paper refers to a subset of strategic games within game theory. A game is defined to have $P$ players and $M$ moves per player. Each player can play any one of their $M$ moves every round, and scores for each player are calculated by accessing the payoff matrix with the results of all their opponents. The payoff matrix is represented by a $P$-dimensional hypercube matrix with side length $M$. Players can be conceptualized as being arranged in a circle, each player is the first in their perspective, which increments for player to their right. Thus, player $P_i$'s' $(1\leq i\leq P)$ score is calculated by accessing the payoff matrix with the coordinates $(P_i, P_{i+1}, \cdots , P_P, P_1, P_2, \cdots , P_{i-1})$. Figure \ref{fig:1} shows an example of how payoffs could be calculated for each player. Players $1$ and $2$ make moves $P_1$ and $P_2$, respectively. Thus, the players will recieve scores $M_{<1,2>}=3$ and $M_{<2,1>}=3$, respectively.

Our project utilizes neural networks and natural selection to optimize strategies for mixed strategy games and maximize individual returns. By providing neural networks solely with information on the moves of each player for the last $N$ rounds, and using natural selection based on the resultant score after a number of rounds with different players, we can observe the strategies emerging in the best neural networks, and test how they react to certain situations. We were particularly keen on observing more complicated payoff matrices, such as when $P>2$, $M>2$, as well as games similar to the Prisoner's Dilemma, where nash equilibrium exists, but is not necessarily optimal in the long term.

\section{Methods and Procedure}

\section{Results}

\section{Discussion}

\end{document}